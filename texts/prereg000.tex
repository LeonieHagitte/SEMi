% Options for packages loaded elsewhere
\PassOptionsToPackage{unicode}{hyperref}
\PassOptionsToPackage{hyphens}{url}
\documentclass[
  man,
  american]{apa7}
\usepackage{xcolor}
\usepackage{amsmath,amssymb}
\setcounter{secnumdepth}{-\maxdimen} % remove section numbering
\usepackage{iftex}
\ifPDFTeX
  \usepackage[T1]{fontenc}
  \usepackage[utf8]{inputenc}
  \usepackage{textcomp} % provide euro and other symbols
\else % if luatex or xetex
  \usepackage{unicode-math} % this also loads fontspec
  \defaultfontfeatures{Scale=MatchLowercase}
  \defaultfontfeatures[\rmfamily]{Ligatures=TeX,Scale=1}
\fi
\usepackage{lmodern}
\ifPDFTeX\else
  % xetex/luatex font selection
\fi
% Use upquote if available, for straight quotes in verbatim environments
\IfFileExists{upquote.sty}{\usepackage{upquote}}{}
\IfFileExists{microtype.sty}{% use microtype if available
  \usepackage[]{microtype}
  \UseMicrotypeSet[protrusion]{basicmath} % disable protrusion for tt fonts
}{}
\makeatletter
\@ifundefined{KOMAClassName}{% if non-KOMA class
  \IfFileExists{parskip.sty}{%
    \usepackage{parskip}
  }{% else
    \setlength{\parindent}{0pt}
    \setlength{\parskip}{6pt plus 2pt minus 1pt}}
}{% if KOMA class
  \KOMAoptions{parskip=half}}
\makeatother
\usepackage{graphicx}
\makeatletter
\newsavebox\pandoc@box
\newcommand*\pandocbounded[1]{% scales image to fit in text height/width
  \sbox\pandoc@box{#1}%
  \Gscale@div\@tempa{\textheight}{\dimexpr\ht\pandoc@box+\dp\pandoc@box\relax}%
  \Gscale@div\@tempb{\linewidth}{\wd\pandoc@box}%
  \ifdim\@tempb\p@<\@tempa\p@\let\@tempa\@tempb\fi% select the smaller of both
  \ifdim\@tempa\p@<\p@\scalebox{\@tempa}{\usebox\pandoc@box}%
  \else\usebox{\pandoc@box}%
  \fi%
}
% Set default figure placement to htbp
\def\fps@figure{htbp}
\makeatother
\setlength{\emergencystretch}{3em} % prevent overfull lines
\providecommand{\tightlist}{%
  \setlength{\itemsep}{0pt}\setlength{\parskip}{0pt}}
\usepackage[style=apa,]{biblatex}
\addbibresource{bibliography.bib}
\usepackage{csquotes}
\usepackage{amsmath, amssymb, bm}
\usepackage{graphicx}
\usepackage{subcaption}
\usepackage{caption}
\usepackage{geometry}
\usepackage{float}
\usepackage{rotating}

\captionsetup{justification=raggedright,singlelinecheck=false}

\usepackage[english]{babel}

\title{\textbf{\textsf{Preregistration: SEMi-Complete by Design: \\ A Monte Carlo simulation to assess Measurement Invariance in Moderated Nonlinear Factor Analysis and SEM Trees}}}
\date{\today}
\author{Leonie Hagitte\textsuperscript{1,2}, Andreas M. Brandmaier\textsuperscript{2}}

% math
\usepackage{amsmath, amssymb}

% references
\usepackage[style=apa, backend=biber]{biblatex}
\addbibresource{bibliography.bib}

% fonts
\usepackage{helvet}
\usepackage{sectsty}
\allsectionsfont{\sffamily} % for section titltes use sans-serif
% \renewcommand{\familydefault}{\sfdefault} % comment out for sans-serif font
% \usepackage{sansmath} % comment out for sans-serif math font
% \sansmath % comment out for sans-serif math font

% margins
\usepackage{geometry}
\geometry{
  a4paper,
  total={170mm,257mm},
  left=25mm,
  right=25mm,
  top=30mm,
  bottom=30mm,
}

% no indentation when a new paragraph starts
\setlength{\parindent}{0cm}

% links
\usepackage{hyperref} % better links
\usepackage{color}    % nicer link colors
\definecolor{pigment}{rgb}{0.2, 0.2, 0.6}
\hypersetup{
  colorlinks = true, % Color links instead of ugly boxes
  urlcolor   = pigment, % Color for external hyperlinks
  linkcolor  = black, % Color for internal links
  citecolor  = pigment % Color for citations
}

% headers
\usepackage{fancyhdr}
\pagestyle{fancy}
\lhead{ADEMP-PreReg MNLFA vs. SEM-Trees}
\chead{}
\rhead{}

% example boxes
\usepackage{tcolorbox}
\newtcolorbox{examplebox}{
  colback=white,
  colframe=gray!30,
  title=Example,
  sharp corners,
  boxrule=0.5pt,
  coltitle=black
}

% conditionals
\usepackage{ifthen}
\newboolean{showinstructions}
\newboolean{showexamples}
\newboolean{showexplanations}
\renewenvironment{examplebox}{%
  \ifthenelse{\boolean{showexamples}}%
    {\begin{tcolorbox}[colback=white, colframe=gray!30, title=Example, sharp corners, boxrule=0.5pt, coltitle=black]}%
    {\expandafter\comment}%
}{%
  \ifthenelse{\boolean{showexamples}}%
    {\end{tcolorbox}}%
    {\expandafter\endcomment}%
}

% Define a new environment for explanations
\newcommand{\explanation}[1]{%
  \ifthenelse{\boolean{showexplanations}}%
    {\textit{Explanation:} #1}%
    {\ignorespaces}%
}

% Define a new environment for instructions
\newcommand{\instructions}[1]{%
  \ifthenelse{\boolean{showinstructions}}%
    {#1}%
    {\ignorespaces}%
}

\makeatletter
\newcommand{\maketitlepage}{%
    \begin{titlepage}
        \maketitle
        \thispagestyle{empty}
        \vfill 
        \centering
        Version: 1.1 \\
        Last updated: 2024-11-18 \\
        Template based on: \\
        Siepe, B. S., Bartoš, F., Morris, T. P., Boulesteix, A.-L., Heck, D. W., \& Pawel, S. (2024). Simulation Studies for Methodological Research in Psychology: A Standardized Template for Planning, Preregistration, and Reporting. \textit{Psychological Methods}. \url{https://doi.org/10.1037/met0000695}, \url{https://doi.org/10.31234/osf.io/ufgy6}\\
        \vspace{\baselineskip}
        \textsuperscript{1}Center for Lifespan Psychology, Max Planck Institute for Human Development, Berlin, Germany; The International Max Planck Research School on the Life Course, Berlin, Germany\\
        \vspace{\baselineskip}
        \textsuperscript{2}Department of Psychology, MSB Medical School Berlin, Berlin, Germany
        \vfill 
    \end{titlepage}
    \newpage
}
\makeatother

\setboolean{showinstructions}{true}
\setboolean{showexamples}{true}
\setboolean{showexplanations}{true}

\usepackage[style=apa, sortcites=true, sorting=nyt, backend=biber]{biblatex}
\DeclareLanguageMapping{american}{american-apa}
\addbibresource{bibliography.bib}
\usepackage{bookmark}
\IfFileExists{xurl.sty}{\usepackage{xurl}}{} % add URL line breaks if available
\urlstyle{same}
\hypersetup{
  pdftitle={SEMi-Complete by Design: A Monte Carlo simulation to assess Measurement Invariance in Moderated Nonlinear Factor Analysis and SEM-Trees},
  pdfauthor={Leonie Hagitte; Andreas M. Brandmaier},
  hidelinks,
  pdfcreator={LaTeX via pandoc}}

\title{SEMi-Complete by Design: A Monte Carlo simulation to assess
Measurement Invariance in Moderated Nonlinear Factor Analysis and
SEM-Trees}
\author{Leonie Hagitte \and Andreas M. Brandmaier}
\date{2025-12-10}

\begin{document}
\maketitle

\subsection{General Information}\label{general-information}

\subsubsection{What is the title of the
project?}\label{what-is-the-title-of-the-project}

SEMi-Complete by Design: A Monte Carlo simulation to assess Measurement
Invariance in Moderated Nonlinear Factor Analysis and SEM-Trees.

\subsubsection{Who are the current and future project
contributors?}\label{who-are-the-current-and-future-project-contributors}

Leonie Hagitte, Andreas M. Brandmaier

\subsubsection{Provide a description of the
project.}\label{provide-a-description-of-the-project.}

Ensuring the validity of psychological assessments is crucial, yet
differential item functioning (DIF) can threaten measurement invariance
(MI) when test items function differently across groups
\autocite{Bauer2020}. Recent calls for improved DIF detection methods
emphasize the need for more advanced statistical approaches
\autocite{Lee2024}. Moderated nonlinear factor analysis (MNLFA) is a
recent approach for assessing MI via parameter moderation within a
single-group confirmatory factor analysis framework. MLNFA evaluates MI
across multiple continuous and categorical covariates, and accounts for
heteroskedasticity by modeling factor and residual variances as
functions of these covariates. While MNLFA offers continuous moderation
of several parameters of SEMs (e.g.: factor loadings, covariances etc.),
it requires a priori specification of covariates and their functional
relationships \autocite{Bauer2017,Kolbe2024}. In contrast, structural
equation modeling (SEM) trees and forests are data-driven,
non-parametric methods that use recursive partitioning to identify
latent subgroups in which model parameters differ, without assuming
specific functional forms or predefined covariate effects. These
approaches allow for nonlinear moderation of factor loadings and can
reveal complex interaction effects, enabling the exploratory detection
of DIF \autocite{Brandmaier2016,Brandmaier2013}. In this study, we
conducted a Monte Carlo simulation to compare the performance of MNLFA
and SEM trees and forests in detecting DIF and assessing MI under
varying conditions. Specifically, we evaluate their effectiveness in
identifying non-invariance and detecting relevant covariates. Our
findings will inform best practices for selecting statistical techniques
to test MI in psychological assessment.

\subsubsection{Did any of the contributors already conduct related
simulation studies on this specific
question?}\label{did-any-of-the-contributors-already-conduct-related-simulation-studies-on-this-specific-question}

Neither of the authors has conducted simulation studies that are of
immediate relevance to the current project before. However, Andreas
Brandmaier was involved in conducting simulation studies on SEM trees/
SEM forests as well as SEM modeling in general
e.g.\autocite{Buchberger2024,SilvaDíaz2025,Arnold2021}.

\subsection{Aims}\label{aims}

\subsubsection{What is the aim of the simulation
study?}\label{what-is-the-aim-of-the-simulation-study}

The aim of this simulation study is to evaluate different methods
(i.e.~MNLFA and SEM trees/ SEM forests) regarding their accuracy, power,
and false-positive rate in detecting MI across different factor models
(ADEMP category `hypothesis testing').

\subsection{Data-Generating Mechanism}\label{data-generating-mechanism}

\subsubsection{How will the parameters for the data-generating mechanism
(DGM) be
specified?}\label{how-will-the-parameters-for-the-data-generating-mechanism-dgm-be-specified}

The data will be generated parametrically. Different population
structural equation models (SEM) with latent variables and continuous
indicators will be simulated in two different simulation studies:

\subsubsection{Study 1}\label{study-1}

\textbf{Single-Factor Model with Moderated Parameters}

We specify a confirmatory factor analysis model (model 1) with a single
latent factor \(\eta\) and four observed indicators
\(\mathbf{x} = (x_1, \dots, x_4)^\top\). The influence of a continuous
moderator is introduced into selected parameters via predefined
transformation functions applied to an underlying raw moderator
variable. The population model is:

\[
\mathbf{x} = \boldsymbol{\nu}(Z) + \boldsymbol{\Lambda}_x(Z)\,\eta + \boldsymbol{\varepsilon},
\qquad
\boldsymbol{\varepsilon} \sim \mathcal{N}(\mathbf{0}, \boldsymbol{\Theta}_\varepsilon).
\]

where

\begin{itemize}
\tightlist
\item
  \(\boldsymbol{\nu}(Z)\) is the vector of intercepts,
\item
  \(\boldsymbol{\Lambda}_x(Z)\) is the \(4 \times 1\) factor loading
  matrix,
\item
  \(\boldsymbol{\varepsilon} \sim \mathcal{N}\!\left(\mathbf{0}, \boldsymbol{\Theta}_\varepsilon\right)\)
  is the residual vector,
\item
  \(\eta \sim \mathcal{N}\!\left(\mu_\eta(Z), \sigma^2_\eta(Z)\right)\)
  is the latent factor,
\item
  \(Z = h(M)\) is the effective moderator entering the model equations,
  obtained by transforming a raw continuous covariate \(M\).
\end{itemize}

\paragraph{Moderator Transformation}\label{moderator-transformation}

Let \(M \sim \mathcal{U}(-1,1)\) denote a bounded continuous covariate.

The effective moderator is defined as \[
Z = h(M),
\]

where \[h(\cdot)\] is chosen from the following deterministic
transformation functions. Each transformation type is treated as a
distinct condition in the simulation design:

\begin{enumerate}
\def\labelenumi{\arabic{enumi}.}
\item
  \textbf{Linear:} \[
  h(M) = M
  \] In this case, the effective moderator is bounded to the interval
  \((-1,1)\) and symmetric around zero.
\item
  \textbf{Sigmoid:} \[
  h(M) = a + \frac{b-a}{1 + \exp\!\bigl(-k(M-c)\bigr)},
  \qquad a=-1,\; b=1,\; c=0,\; k>0.
  \] This transformation maps the input to the open interval \((-1, 1)\)
  and yields a bounded moderator symmetric around zero, exhibiting
  diminishing sensitivity at extreme values.
\item
  \textbf{Quadratic:} \[
  h(M) = 2M^2 - 1
  \] This transformation captures symmetric nonlinear moderation as a
  function of distance from the center while remaining bounded in
  \((-1,1)\).
\item
  \textbf{Noise:} \[
  h(M) = 0
  \] In this condition, the moderator has no effect on any model
  parameter and serves as a non-informative (noise) moderator.
\end{enumerate}

Null moderation is represented by the noise-moderator condition
\((Z=0)\), not by \(\delta=0\); moderation coefficients are held fixed
across conditions. All parameter-level moderation effects are specified
as linear functions of the effective moderator \(Z\). Nonlinearity in
moderation arises exclusively through the transformation \(Z = h(M)\),
not through nonlinear parameter functions.

\%\%\%\% Figure \%\%\%\%\%\%\%

\paragraph{Factor Loadings}\label{factor-loadings}

The baseline factor loadings were fixed to a discrete value of
\(\lambda_{xi} = 0.7\).\\
Variation in loadings across simulation conditions arises exclusively
through the presence or absence of moderation, which is treated as a
fully crossed factor in the grid design.

\textbf{Moderated items:}

At the population level, the baseline (unmoderated) measurement model is
specified with a single latent factor and four indicators. The factor
loading matrix is constructed such that all four indicators share a
common loading value \(\lambda\):

\[
\boldsymbol{\Lambda}_x =
\begin{bmatrix}
\lambda_1 \\
\lambda_2 \\
\lambda_3 \\
\lambda_4
\end{bmatrix}
\] The latent factor variance is fixed at a prespecified constant value
\(\psi_{\eta}= 1\) and is held constant across all simulation
conditions.

For each condition involving moderation, the loading of item \(i\) is
defined as a deterministic function of the moderator \(Z\):

\[
\lambda_{xi}(Z) = \lambda_{xi} + \delta_{\lambda} Z,
\quad Z \in [-1, 1].
\]

The parameter \(\delta_{\lambda}\) represents the strength of the
moderation effect and is specified using a discrete set of admissible
values chosen to ensure that the moderated loading remains within a
psychometrically plausible interval:

\[
\lambda_{xi}(Z) \in [0.3,\, 1.0].
\]

Given the fixed baseline loading \(\lambda_{xi}=0.7\), this constraint
yields an admissible range \[
\delta_{\lambda} \in [-0.4,\, 0.3].
\]

\(\delta_{\lambda}\) is evaluated at the following discrete levels:

\[
\delta_{\lambda} \in \{-0.4,\,-0.2,\,0.2,\,0.3\}.
\]

For moderator transformations beyond the linear case (quadratic or
sigmoid), the functional form of \(\lambda_{xi}(Z)\) is adapted
accordingly, and the same discrete set of admissible moderation
strengths is applied. This yields fully crossed combinations of (i) type
of moderator transformation, (ii) number of moderated items, and (iii)
moderation strength.

\textbf{Factor loading matrix:}

\[
\boldsymbol{\Lambda}_x(Z) =
\begin{bmatrix}
\lambda_1(Z) \\
\lambda_2(Z) \\
\lambda_3(Z)\\
\lambda_4(Z)
\end{bmatrix}
\]

Under partial moderation, only \(\lambda_1(Z)\) and \(\lambda_2(Z)\)
depend on \(Z\), while \(\lambda_3(Z)=\lambda_3\) and
\(\lambda_4(Z)=\lambda_4\). Under full moderation, all four loadings
vary with \(Z\).

\begin{itemize}
\item
  \textbf{Null Model:} Neither factor loadings, nor intercepts are
  moderated via Z. There is no moderation present in this DGP.
\item
  \textbf{Full moderation (Model 1.1):} All factor loadings and all
  intercepts vary with \(Z\): \[
    \lambda_{xi}(Z) = \lambda_{xi} + \delta_{\lambda} Z, 
    \qquad
    \nu_{i}(Z) = \nu_{i} + \delta_{\nu} Z,
  \]
\item
  \textbf{Partial moderation (Model 1.2):}
\item
  Only the loadings of items \(1\) and \(2\) and their intercepts can be
  moderated by Z: \[
    \lambda_{x1}(Z) = \lambda_{x1} + \delta_{\lambda} Z,
    \qquad
    \lambda_{x2}(Z) = \lambda_{x2} + \delta_{\lambda} Z,
  \] \[
    \nu_{x1}(Z) = \nu_{x1} + \delta_{\nu} Z,
    \nu_{x2}(Z) = \nu_{x2} + \delta_{\nu} Z,
  \] whereas all remaining loadings and intercepts as well as residual
  variances \((\theta_{i0})\) remain fixed at their baseline values.
\end{itemize}

\paragraph{Residual Variances}\label{residual-variances}

Item reliabilities are varied deterministically across simulation
conditions using the fixed grid. \[
Reliability_i \in \{0.60,\,0.70,\,0.80,\,0.95\}
\]

Baseline residual variances are derived analytically from a fixed latent
variance \(\psi_\eta=\psi_0>0\) and indicator-specific reliability
values: \[
Reliability_i=\frac{\lambda_{xi}^2 \psi_0}{\lambda_{xi}^2 \psi_0 + \theta_{i0}}\quad \Rightarrow \quad\theta_{i0}=\frac{\lambda_{xi}^2 \psi_0(1-\text{Reliability}_i)}
     {\text{Reliability}_i}.
\] \#\#\#\# Intercepts

Intercept moderation is manipulated via two fully crossed factors:\\
(i) the pattern of moderated items and\\
(ii) the magnitude of the moderation effect. For each indicator \(i\),
the intercept is modeled as \[
\nu_i(Z) = \nu_i + \delta_{\nu} Z,
\] where \(\nu_i\) denotes the baseline intercept and \(\delta_{\nu}\)
controls the strength and direction of intercept moderation. The pattern
of moderated items is varied across three levels:

\begin{itemize}
\item
  \textbf{No intercept moderation:} \(\delta_{\nu} = 0\) for all
  \(i = 1,\dots,4\), such that \[
  \nu_i(Z) = \nu_i \quad \text{for all } i.
  \]
\item
  \textbf{Partial intercept moderation (items 1 and 2):}\\
  \[
    \delta_{\nu} \in \{-1.0,\,-0.5,\,0.5,\,1.0\} \quad \text{for } i \in \{1,2\},
    \qquad
    \delta_{\nu} = 0 \quad \text{for } i \in \{3,4\},
    \] such that only \(\nu_1(Z)\) and \(\nu_2(Z)\) vary with \(Z\).
\item
  \textbf{Full intercept moderation (all items):}\\
  \[
    \delta_{\nu} \in \{-1.0,\,-0.5,\,0.5,\,1.0\} \quad \text{for } i = 1,\dots,4,
  \] such that the intercepts of all four indicators vary with \(Z\).
\end{itemize}

Baseline intercepts are varied deterministically across simulation
conditions using the fixed grid \[
\nu_i \in \mathcal{G}_\nu = \{-1,\, 0,\, 1\},
  \] providing low, medium, and high intercept levels centered around
zero. The simulation design thus yields fully crossed combinations of
(a) intercept moderation pattern (none, partial, full), (b) moderation
strength levels \(\delta_{\nu}\), and (c) baseline intercept levels
\(\nu\).

\paragraph{Latent Distribution}\label{latent-distribution}

Latent mean moderation was specified deterministically as: \[
\mu_\eta(Z) = \alpha + \delta_\eta Z,
\]

with both parameters selected from discrete sets:

\[
\alpha \in \{0\}, \qquad
\delta_\eta \in \{-1.0,\,-0.5,\,0.5,\,1.0\}.
\]

The latent variance was modeled to be fixed at 1.

\pagebreak

\paragraph{Analytical Model Study 1}\label{analytical-model-study-1}

\begin{center}\includegraphics[width=0.7\linewidth]{tikz_model1} \end{center}

The analytical model will also have one latent factor \(\eta_1\). With
four manifest indicators \({x_1,..., x_4}\), their factor loadings
\({\lambda_1,..., \lambda_4}\) and their respective residuals
\({\epsilon_1,..., \epsilon_4}\). In the analytical model the residuals
will be assumed and modeled to be uncorrelated. Furthermore, the model
includes a moderator variable \(Z\), which can moderate the intercepts
and factor loadings from none, one or two of the manifest variables. The
moderator variable \(Z\) will be modeled as linear or quadratic.

\%\%\%\%\%\%\%\%\%\%\%\%\%\%\%\%\%\%\%\%\%\%\%\%\%\%\%\%\%\%\%\%\%\%\%\%\%\%\%\%\%\%\%\%\%\%\%\%\%\%\%\%\%\%\%\%\%\%\%\%\%\%\%\%\%\%\%\%\%\%\%\%
\#\#\# Study 2

\textbf{Two Factors with Moderator Effects} We specify a confirmatory
factor analysis model with two latent factors, \(\eta_1\) and
\(\eta_2\), and seven observed indicators
\(\mathbf{x} = (x_1, \dots, x_4)^\top\) and
\(\mathbf{y} = (y_1, \dots, y_3)^\top\). Along their respective
residuals \({\epsilon_1,...,\epsilon_4}\) and
\({\delta_1,..., \delta_3}\). The first factor loads onto indicators
\(x_1\) to \(x_4\), the second onto \(y_1\) to \(y_3\). A continuous
moderator variable \(Z\) introduces conditional heterogeneity into
selected measurement parameters (i.e.~intercepts, loadings and latent
factor covariance) using one of three functional forms (linear,
quadratic or sigmoid).

\[
\mathbf{x} = \boldsymbol{\nu}_x(Z) + \boldsymbol{\Lambda}_x(Z)\,\eta_1 + \boldsymbol{\varepsilon}(Z),
\] \[
\mathbf{y} = \boldsymbol{\nu}_y(Z) + \boldsymbol{\Lambda}_y(Z)\,\eta_2 + \boldsymbol{\delta}(Z),
\]

where:

\begin{itemize}
\tightlist
\item
  \(\boldsymbol{\nu}_x(Z)\) and \(\boldsymbol{\nu}_y(Z)\) are the
  \(4 \times 1\) and \(3 \times 1\) vectors of intercepts for
  \(\mathbf{x}\) and \(\mathbf{y}\), respectively,
\item
  \(\boldsymbol{\Lambda}_x(Z)\) is the \(4 \times 1\) loading vector for
  \(\eta_1\), and \(\boldsymbol{\Lambda}_y(Z)\) is the \(3 \times 1\)
  loading vector for \(\eta_2\),
\item
  \(\boldsymbol{\varepsilon}(Z) \sim \mathcal{N}\bigl(\mathbf{0}, \boldsymbol{\Theta}_x(Z)\bigr)\)
  and
  \(\boldsymbol{\delta}(Z) \sim \mathcal{N}\bigl(\mathbf{0}, \boldsymbol{\Theta}_y(Z)\bigr)\)
  are the residual vectors for \(\mathbf{x}\) and \(\mathbf{y}\),
\item
  \(\boldsymbol{\eta} = (\eta_1, \eta_2)^\top \sim \mathcal{N}\bigl(\boldsymbol{\mu}_\eta(Z), \boldsymbol{\Phi}_\eta(Z)\bigr)\)
  is the latent factor vector,
\item
  \(Z = h(M)\) is the effective moderator entering the model equations,
  obtained by transforming a raw continuous covariate \(M\).
\end{itemize}

\paragraph{Moderator Transformation}\label{moderator-transformation-1}

Let \(M \sim \mathcal{U}(-1,1)\) denote a bounded continuous covariate.

The effective moderator is defined as \[
Z = h(M),
\]

where \[h(\cdot)\] is chosen from the following deterministic
transformation functions. Each transformation type is treated as a
distinct condition in the simulation design:

\begin{enumerate}
\def\labelenumi{\arabic{enumi}.}
\item
  \textbf{Linear:} \[
  h(M) = M
  \] In this case, the effective moderator is bounded to the interval
  \((-1,1)\) and symmetric around zero.
\item
  \textbf{Sigmoid:} \[
  h(M) = a + \frac{b-a}{1 + \exp\!\bigl(-k(M-c)\bigr)},
  \qquad a=-1,\; b=1,\; c=0,\; k>0.
  \] This transformation maps the input to the open interval \((-1, 1)\)
  and yields a bounded moderator symmetric around zero, exhibiting
  diminishing sensitivity at extreme values.
\item
  \textbf{Quadratic:} \[
  h(M) = 2M^2 - 1
  \] This transformation captures symmetric nonlinear moderation as a
  function of distance from the center while remaining bounded in
  \((-1,1)\).
\item
  \textbf{Noise:} \[
  h(M) = 0
  \]
\end{enumerate}

In this condition, the moderator has no effect on any model parameter
and serves as a non-informative (noise) moderator.

Null moderation is represented by the noise-moderator condition
\((Z=0)\), not by \(\delta=0\); moderation coefficients are held fixed
across conditions. All parameter-level moderation effects are specified
as linear functions of the effective moderator \(Z\). Nonlinearity in
moderation arises exclusively through the transformation \(Z = h(M)\),
not through nonlinear parameter functions.

\paragraph{Factor Loadings}\label{factor-loadings-1}

For each latent factor, all baseline loadings for that factor are fixed
at 0.7. For the two-factor model in Study\textasciitilde2, this yields

\[
\boldsymbol{\Lambda}_x =
\begin{bmatrix}
0.7\\
0.7\\
0.7\\
0.7
\end{bmatrix},
\qquad
\boldsymbol{\Lambda}_y =
\begin{bmatrix}
0.7\\
0.7\\
0.7
\end{bmatrix},
\]

Moderation of factor loadings is introduced only for the first two
\(x\)-indicators. For each moderated indicator \(i \in \{1,2\}\), \[
\lambda_{xi}(Z) = \lambda_{xi} + \delta_{\lambda} Z .
\]

To ensure that moderated loadings remain within an admissible interval,
\[
\lambda_{xi}(Z) \in [0.3,\,1.0],
\]

the moderation coefficients are restricted to \[
\delta_{\lambda} \in [-0.4,\,0.3],
\]

and evaluated on the discrete grid (leaving out the zero effect) \[
\delta_{\lambda} \in \{-0.4,\,-0.2,\,0.2,\,0.3\}.
\]

Thus, the loading matrices take the form \[
\boldsymbol{\Lambda}_x(Z)=
\begin{bmatrix}
\lambda_{x1}(Z)\\
\lambda_{x2}(Z)\\
0.7\\
0.7
\end{bmatrix},
\qquad
\boldsymbol{\Lambda}_y(Z)=
\begin{bmatrix}
0.7\\
0.7\\
0.7
\end{bmatrix},
\]

yielding fully crossed combinations of (i) moderator transformation type
\(Z=h(M)\), (ii) moderated vs.~unmoderated indicators, and (iii)
moderation strength \(\delta_{\lambda}\).

\paragraph{Residual Variances}\label{residual-variances-1}

Item reliabilities are varied deterministically across simulation
conditions using the fixed grid. \[
Reliability_i \in \{0.60,\,0.70,\,0.80,\,0.95\}
\]

Baseline residual variances are derived analytically from a fixed latent
variance \(\psi_\eta=\psi_0>0\) and indicator-specific reliability
values:

\[
Reliability_i=\frac{\lambda_{xi}^2 \psi_0}{\lambda_{xi}^2 \psi_0 + \theta_{i0}}\quad \Rightarrow \quad\theta_{i0}=\frac{\lambda_{xi}^2 \psi_0(1-\text{Reliability}_i)}
     {\text{Reliability}_i}.
\] \#\#\#\# Latent Variable Distribution

Latent means are held constant at zero for both latent factors:

\[
\boldsymbol{\mu}_\eta(Z) = 
\begin{bmatrix}
0 \\[4pt]
0
\end{bmatrix}.
\]

Latent variances are allowed to vary with the effective moderator \(Z\)
through the log-linear specification

\[
\sigma^2_{\eta_j}(Z)
  = \exp\!\left(\beta_{0j} + \beta_{1j} Z\right),
\qquad j = 1,2,
\] ensuring positivity for all variance values.

The baseline variance is set to 1 and moderation strength is selected
from discrete grids:

\[
\delta_{\sigma_\eta} \in \{-0.4,\,-0.2,\,0.2,\,0.3\}.
\]

The latent covariance between \(\eta_1\) and \(\eta_2\) is fixed across
all conditions:

\[
\phi_{12}(Z) = 0.4,
\]

yielding the latent covariance matrix \[
\boldsymbol{\Phi}_\eta(Z)
=
\begin{bmatrix}
\sigma^2_{\eta_1}(Z) & 0.4 \\
0.4 & \sigma^2_{\eta_2}(Z)
\end{bmatrix}.
\]

\%\%\%\%\%\%\%\%\%\%\%\%\%\%\%\%\%\%\%\%\%\%\%\%\%\%\%\%\%\%\%\%\%\%\%\%\%\%\%\%\%\%\%\%\%\%\%\%\%\%\%\%

\{\small \% reduce font size for figure caption only

\begin{sidewaysfigure}
\noindent{\textbf{Analytical Models Study 2}}
\centering

\begin{subfigure}[b]{0.47\textwidth}
    \centering
    \includegraphics[width=\linewidth]{tikz_model2.0.png}
    \caption{Model 2.0}
\end{subfigure}
\hspace{0.3cm} % tighter space
\begin{subfigure}[b]{0.47\textwidth}
    \centering
    \includegraphics[width=\linewidth]{tikz_model2.1.png}
    \caption{Model 2.1}
\end{subfigure}

\vspace{0.3cm} % reduced vertical space

\begin{subfigure}[b]{0.47\textwidth}
    \centering
    \includegraphics[width=\linewidth]{tikz_model2.2.png}
    \caption{Model 2.2}
\end{subfigure}
\hspace{0.3cm} % tighter space
\begin{subfigure}[b]{0.47\textwidth}
    \centering
    \includegraphics[width=\linewidth]{tikz_model2.3.png}
    \caption{Model 2.3}
\end{subfigure}

\caption{We specify a confirmatory factor analysis model with two latent factors, $\eta_1$ and $\xi_1$, and seven observed indicators, $\mathbf{x} = (x_1, \dots, x_4)^\top$ and $\mathbf{y} = (y_1, y_2, y_3)^\top$, along with their respective residuals, $\varepsilon_1, \dots, \varepsilon_4$ and $\delta_1, \delta_2, \delta_3$. The first factor, $\eta_1$, loads onto indicators $x_1$ to $x_4$, while the second factor, $\xi_1$, loads onto indicators $y_1$ to $y_3$. A continuous moderator variable $Z$ introduces conditional heterogeneity into selected measurement parameters, including residuals, intercepts, factor loadings, and latent factor covariance, using one of two functional forms (linear or quadratic). Red dashed arrows indicate model misspecifications.}
\end{sidewaysfigure}

\}

\pagebreak

\subsection{What will be the different factors of the data-generating
mechanism?}\label{what-will-be-the-different-factors-of-the-data-generating-mechanism}

\%\explanation{A factor can be a parameter/setting/process/etc. that determines the data-generating mechanism and is varied across simulation conditions.}

The simulation design includes several manipulated factors that govern
the data-generating mechanism (DGM). These factors are varied across
simulation conditions and apply consistently to both studies, unless
otherwise noted.

\subsubsection*{Core DGM Factors (Applied in Both Studies)}

The following factors are systematically varied in both studies using a
grid search design:

\begin{itemize}
    \item \textbf{Functional form of moderation:} The moderator variable \( Z \) is defined as a transformation of \( X \in [-1, 1] \), evaluated systematically using three functional forms:
    \begin{itemize}
        \item Linear: \( Z = X \)
        \item Quadratic: \( Z = X^2 \)
        \item Sigmoid: \( Z = \frac{1}{1 + \exp(-\delta(X - c))} \)
    \end{itemize}
    For the sigmoid function, parameters are fixed to:
    \[
    c = 0, \quad \delta \in \{5, 2.5, 1\}
    \]
    representing steep, moderate, and relatively flat sigmoid curves, respectively.
    
    \item \textbf{Moderated parameters:} The parameters subject to moderation by \( Z \) are systematically specified across grid conditions. For each condition, moderation is applied to:
    \begin{itemize}
        \item None
        \item One parameter (e.g., \(\lambda_{x1}\) or \(\nu_1\))
        \item Two parameters (e.g., \(\lambda_{x1}, \lambda_{x2}\))
    \end{itemize}
    This replaces random selection with explicit condition specification for clear, reproducible simulation configurations.

   \item \textbf{Latent variance moderation:} The latent variance is modeled as a log-linear function of \( Z \):  
    \[
    \sigma^2_\eta(Z) = \exp(\beta_0 + \beta_1 Z)
    \]
    where:
    \[
    \beta_0 \in \{-0.5, 0, 0.5\}, \quad \beta_1 \in \{-0.5, 0, 0.5\}
    \]
    \item \textbf{Model misspecification of moderator form:} The DGM defines \( Z \) using one of the three functional forms above, while the analytical model assumes either a \textit{linear} or \textit{quadratic} moderator effect.
\end{itemize}

\subsubsection*{Study-Specific Factors}

\paragraph{Study 1.}

Focuses on functional form misspecification and data-related conditions:

\begin{itemize}
    \item \textbf{Analysis model:} Assumes either a linear or quadratic moderator.
    \item \textbf{Sample sizes:} \(N \in \{300, 500, 700, 1000\}\) 
    \item \textbf{Indicator reliabilities:} Low (.6), moderate (.7), moderate to high (.8), or high (.95).
    \item \textbf{No structural misspecification:} The factor structure and residual terms follow the specified model.
\end{itemize}

\paragraph{Study 2.}

Builds on Study 1 and introduces additional model misspecification in
the factor structure:

\begin{itemize}
    \item \textbf{Same DGM and moderator form variation as in Study 1.}
    \item \textbf{Analysis model:} Assumes either a linear or quadratic form for the moderator.
    \item \textbf{Structural model misspecifications:} (between-subjects factor)
    \begin{enumerate}
        \item \textbf{No misspecification:} Factor structure matches the analysis model.
        \item \textbf{Cross-loadings:} Items \( y_5 \), \( y_6 \) additionally load on a second latent factor:  
        \[
        \lambda^{(CL)}_5, \lambda^{(CL)}_6 \in \{0.3, 0.4\}
        \]
        \item \textbf{Correlated residuals:} Residual covariances are added between:  
        \[
        \text{Cov}(\varepsilon_1, \varepsilon_5),\quad \text{Cov}(\varepsilon_2, \varepsilon_6) \in \{0.2, 0.3\}
        \]

        \item \textbf{Combined:} Both cross-loadings and residual correlations are present.
    \end{enumerate}
\end{itemize}

\subsection{If there is more than one factor: How will the factor levels be combined and how many simulation conditions will this create?}

We employ a full grid search, in which parameter values are sampled from
pre-specified discrete values. A total of 6000 simulation replications
should ideally be conducted. This number was chosen to ensure that the
confidence intervals for key estimated proportions (e.g., power or Type
I error rates near \(p = 0.8\)) achieve an approximate width of
\(\pm 1\%\). This is based on the calculation: \[
\text{SE} = 2 \times \sqrt{\frac{p(1-p)}{6000}},
\] which yields a standard error consistent with the desired precision
for evaluating simulation outcomes within this study. To ensure
computational feasability, we propose an iterative approach whereby
initial pilot simulations \(n=100\) are conducted to amongst others,
estimate the needed simulation time. These estimates then inform a more
principled determination of the total number of replications, targeting
a predefined precision threshold.

\vspace{\baselineskip}

The factors that vary across simulation conditions include:

\begin{itemize}
    \item The functional form of moderator effects (linear, quadratic, sigmoid)
    \item The type of moderated parameters (e.g., factor loadings, intercepts, residual variances)
    \item The degree of model misspecification (none, cross-loadings, correlated residuals, combined)
    \item Factor loadings (ranging from 0.7 to 0.9)
    \item Item reliabilities (ranging from 0.6 to 0.95)
    \item Moderator strength parameters 
    \item Sample sizes per group (i.e. 300, 500, 700, 1000)
\end{itemize}

\section{Estimands and Targets}
\subsection{What will be the estimands and/or targets of the simulation study?}

\%\explanation{Estimands and other targets jointly refer to the practical aims of the compared methods. For example, an estimand/target could be the true effect size estimated by the compared methods \parencite[see][for more information]{Siepe2024}. Please also specify if some targets are considered more important than others, i.e., if the simulation study will have primary and secondary outcomes.}

\subsection{What will be the estimands and/or targets of the simulation study?}

The primary estimands in this simulation study pertain to the detection
and accurate estimation of measurement non-invariance (MI) due to
moderation effects on measurement parameters. Specifically, we assess
whether the estimation methods compared can correctly identify the
presence or absence of moderation in factor loadings, intercepts, and
residual variances, reflecting the true data-generating structure.
Consequently, key simulation outcomes include Type I error rates and
statistical power associated with detecting such moderation effects, as
well as bias and variability in the recovered parameter estimates.

Secondary estimands include model-level indicators of fit (e.g., RMSEA
with confidence intervals, SRMR, CFI), which serve as auxiliary
diagnostics for methods that yield such metrics (i.e., MNLFA). These are
not applicable to tree-based approaches and are thus interpreted as
method-specific targets rather than general evaluative criteria.

\section{Methods}
\subsection{How many and which methods will be included and which quantities will be extracted?}

\%\explanation{Be as specific as possible regarding the methods that will be compared, and provide a justification for both the choice of methods and their model parameters. This can also include code which will be used to estimate the different methods or models in the simulation with all relevant model parameters. Setting different prior hyperparameters might also be regarded as using different methods. Where package defaults are used, state this. Where they are not used, state what values are used instead.}

We will compare the following methods:

\begin{itemize}
    \item[1)] \textbf{MNLFA} (Moderated Nonlinear Factor Analysis): MNLFA was initially introduced by \textcite{Bauer2009} within the framework of integrative data analysis and was subsequently extended into a general approach for assessing measurement invariance and moderation \parencite{Bauer2017}. The fundamental principle of MNLFA is the parameterization of both measurement and structural model parameters as functions of one or more moderator variables. MNLFA accommodates both frequentist estimation methods, such as maximum likelihood (ML), and Bayesian techniques, including Markov Chain Monte Carlo (MCMC), thereby offering methodological flexibility aligned with specific analytical objectives \parencite{Muench2024}.

    \begin{verbatim}
        %code example?
    \end{verbatim}
        
    \item[2)] \textbf{SEM-Trees}: Structural equation modeling (SEM) trees were originally introduced by \textcite{Brandmaier2013} as a way to systematically search for important covariates and their interactions in data, creating homogeneous subgroups. The method was refined later on by \textcite{Brandmaier2016,Arnold2021}.  SEM trees and forests are data-driven, non-parametric methods that build upon the decision tree paradigm, thus using recursive partitioning to identify latent subgroups in which model parameters differ, without assuming specific functional forms or predefined covariate effects. These approaches allow for nonlinear moderation of factor loadings and can reveal complex interaction effects, enabling the exploratory detection of DIF.

    \begin{verbatim}
        %code example?
    \end{verbatim}

\end{itemize}

For the parametric MNLFA, we will extract estimated moderated parameters
(e.g., factor loadings, intercepts, residual variances), their
associated standard errors, and two-sided \textit{p}-values for testing
the null hypothesis of no moderation effect. Additionally, model fit
indices such as RMSEA (with confidence intervals), SRMR, and CFI will be
recorded.\textbackslash{} For the parameter estimates and fit indices,
we will summarize their distributions across simulation replications by
reporting key quantiles (2.5th, 50th, and 97.5th percentiles). The null
hypothesis will be rejected if the associated \textit{p}-value is less
than the conventional significance level of 0.05. \textbackslash{} In
contrast, non-parametric Tree-based approaches do not yield direct
parameter estimates with standard errors. There detection of MI/
non-invariance is operationalized via the presence of at least one
split. This distinction ensures that each method is evaluated according
to its own inferential framework, while maintaining comparability in
terms of its ability to detect MI reliably.

\section{Performance Measures}
\subsection{Which performance measures will be used?}

\%\explanation{Please provide details on why they were chosen and on how these measures will be calculated. Ideally, provide formulas for the performance measures to avoid ambiguity. Some models in psychology, such as item response theory or time series models, often contain multiple parameters of interest, and their number may vary across conditions. With a large number of estimated parameters, their performance measures are often combined. If multiple estimates are aggregated, specify how this aggregation will be performed. For example, if there are multiple parameters in a particular condition, the mean of the individual biases of these parameters or the bias of each individual parameter may be reported.}

For the MNLFA \textbf{exclusively}, we will evaluate the following
performance measures:

\begin{itemize}
    \item \textbf{Bias}: The average difference between the estimated parameter \(\hat{\theta}\) and the true parameter \(\theta\), defined as
    \[
    \widehat{\text{Bias}} = \frac{1}{n_{\text{sim}}} \sum_{i=1}^{n_{\text{sim}}} \hat{\theta}_i - \theta,
    \]
    where \(n_{\text{sim}}\) is the number of simulation replications.
    
    \item \textbf{Absolute Bias}:
    \[
    \widehat{\text{AbsBias}} = \frac{1}{n_{\text{sim}}} \sum_{i=1}^{n_{\text{sim}}} \left| \hat{\theta}_i - \theta \right|.
    \]
    
    \item \textbf{Relative Bias} (expressed as a proportion of the true parameter):
    \[
    \widehat{\text{RelBias}} = \frac{\widehat{\text{Bias}}}{|\theta|} \quad \text{for } \theta \neq 0.
    \]
    
    \item \textbf{Root Mean Squared Error (RMSE)}:
    \[
    \widehat{\text{RMSE}} = \sqrt{ \frac{1}{n_{\text{sim}}} \sum_{i=1}^{n_{\text{sim}}} (\hat{\theta}_i - \theta)^2 }.
    \]
    
    \item \textbf{Coverage Probability}: The proportion of confidence intervals that contain the true parameter \(\theta\).

    \item \textbf{Model Fit Indices} (e.g., RMSEA) will be reported if applicable, computed according to their standard definitions.
\end{itemize}
\vspace{\baselineskip}

For the SEM trees \textbf{as well as} for the MNLFA we will evaluate:
\vspace{\baselineskip}

\begin{itemize}

    \item \textbf{Type I Error Rate and Power}: The rejection rate of the null hypothesis at significance level \(\alpha = 0.05\), calculated as
    \[
    \widehat{\text{RRate}} = \frac{1}{n_{\text{sim}}} \sum_{i=1}^{n_{\text{sim}}} \mathbf{1}(p_i \leq 0.05),
    \]
    where \(\mathbf{1}(\cdot)\) is the indicator function.
    
\end{itemize}

Performance measures calculated for multiple parameters (e.g., factor
loadings) will be aggregated by reporting the mean value across
parameters.

\subsection{How will Monte Carlo uncertainty of the estimated performance measures be calculated and reported?}

\%\explanation{Ideally, Monte Carlo uncertainty can be reported in the form of Monte Carlo Standard Errors (MCSEs). Please see \textcite{Siepe2024} and \textcite{Morris2019} for a list of formulae to calculate the MCSE related to common performance measures, more accurate jackknife-based MCSEs are available through the \texttt{rsimsum} \parencite{Gasparini2018} and \texttt{simhelpers} \parencite{Simhelpers2022} R packages, the \texttt{SimDesign} \parencite{Chalmers2020} R package can compute confidence intervals for performance measures via bootstrapping. Monte Carlo uncertainty can additionally be visualized using plots appropriate for illustrating variability, such as MCSE error bars, histograms, boxplots, or violin plots of performance measure estimates, if possible (e.g., bias).}

We will quantify Monte Carlo uncertainty using Monte Carlo Standard
Errors (MCSEs) for each performance measure, calculated as follows:

\begin{itemize}
 \item The \textbf{Rejection Rate} refers to the proportion of simulated datasets in which a null hypothesis is rejected, commonly used to estimate empirical Type I error rates or statistical power.\\ The \textbf{MCSE for the Rejection Rate} is calculated assuming a binomial distribution of rejections across \(n_{\text{sim}}\) simulation replications, which reflects the standard error of a proportion estimator:

\[
\text{MCSE}_{\widehat{\text{RRate}}} = \sqrt{\frac{\widehat{\text{RRate}} (1 - \widehat{\text{RRate}})}{n_{\text{sim}}}},
\]

 \item For \textbf{Bias}, MCSE is calculated based on the sample variance of the specific parameter estimates across simulation replications:

\[
\text{MCSE}_{\widehat{\text{Bias}}} = \frac{S_{\hat{\theta}}}{\sqrt{n_{\text{sim}}}},
\]

 \item where $S_{\hat{\theta}} = \sqrt{\sum_{i=1}^{n_{\text{sim}}}{ \{\hat{\theta}_i - (\sum_{i=1}^{n_{\text{sim}}}\hat{\theta}_i/n_{\text{sim}})\}^2}/(n_{\text{sim}} - 1)}$ is the sample standard deviation of the effect estimates.
    \vspace{\baselineskip}

\end{itemize}

\%siepe template as reference \%\textcite{Morris2019} and
\textcite{Whitehead1993}.

\vspace{\baselineskip}

\subsection{How many simulation repetitions will be used for each condition?}

\%\explanation{Please also indicate whether the chosen number of simulation repetitions is based on sample size calculations, on computational constraints, rules of thumb, or any other heuristic or combination of these strategies. Formulas for sample size planning in simulation studies are provided in \textcite{Siepe2024}. If there is a lack of knowledge on a quantity for computing the Monte Carlo standard error (MCSE) of an estimated performance measure (e.g., the variance of the estimator is needed to compute the MCSE for the bias), pilot simulations may be needed to obtain a guess for realistic/worst-case values.}

A total of 6000 simulation replications should ideally be conducted.
This number was chosen to ensure that the confidence intervals for key
estimated proportions (e.g., power or Type I error rates near
\(p = 0.8\)) achieve an approximate width of \(\pm 1\%\). This is based
on the calculation: \[
\text{SE} = 2 \times \sqrt{\frac{p(1-p)}{6000}},
\] which yields a standard error consistent with the desired precision
for evaluating simulation outcomes within this study. To ensure
computational feasability, we propose an iterative approach whereby
initial pilot simulations \(n=100\) are conducted to amongst others,
estimate the needed simulation time. These estimates then inform a more
principled determination of the total number of replications, targeting
a predefined precision threshold.

\subsection{How will missing values due to non-convergence or other reasons be handled?}

\%\explanation{`Convergence' means that a method successfully produces the outcomes of interest (e.g., an estimate, a prediction, a \textit{p}-value, a sample size, etc.) that are required for estimating the performance measures. Non-convergence of some repetitions or whole conditions of simulation studies occurs regularly, e.g., for numerical reasons. It is possible to impute non-converged repetitions, exclude all non-converged repetitions or to implement mechanisms that repeat certain parts of the simulation (such as data generation or model fitting) until convergence is achieved. Further, it is important to consider at which proportion of failed repetitions a whole condition will be excluded from the analysis. See \textcite{Pawel2024} for more guidance on handling missing values in simulation studies.}

Non-convergence or other missing values may occur during model
estimation. We plan to:

\begin{itemize}
    \item Exclude only the specific simulation replicates that fail to converge for a given method while retaining data from other methods within the same replication.
    \item Report the proportion of non-converged cases per method and condition.
    \item If the proportion of non-convergence exceeds a pre-specified threshold (e.g., 5%), we will flag the corresponding simulation condition and interpret results with caution.
\end{itemize}

No imputation of missing estimates will be performed.

\subsection{How do you plan on interpreting the performance measures? \textmd{(optional)}}

\explanation{It can be specified what a `relevant difference' in performance, or what `acceptable' and `unacceptable' levels of performance might be to avoid post-hoc interpretation of performance. Furthermore, some researchers use regression models to analyze the results of simulations and compute effect sizes for different factors, or to assess the strength of evidence for the influence of a certain factor \parencite{Skrondal2000, Chipman2022}. If such an approach will be used, please provide as many details as possible on the planned analyses.}

\begin{examplebox}
We define a type I error rate larger than 5\% as non-acceptable performance. Amongst methods that exhibit acceptable performance regarding the type I error rate (within the MCSE), we consider a method X as performing better than a method Y in a certain simulation condition if the lower bound for the estimated power of method X ($\widehat{\text{Pow}}-\text{MCSE}$) is greater than the upper bound for the estimated power of method Y ($\widehat{\text{Pow}}+\text{MCSE}$).
\end{examplebox}

\textit{Answer:}

\section{Other}
\subsection{Which statistical software/packages do you plan to use?}

\explanation{Likely, not all software used can be prespecified before conducting the simulation. However, the main packages used for model fitting are usually known in advance and can be listed here, ideally with version numbers.}

\begin{examplebox}
We will use the following packages of \texttt{R} version 4.3.1 \parencite{R2020} in their most recent versions: The \texttt{mvtnorm} package \parencite{Genz2009} to generate data, the \texttt{lm()} function included in the \texttt{stats} package \parencite{R2020} to fit the different models, the \texttt{SimDesign} package \parencite{Chalmers2020} to set up  and run the simulation study, and the \texttt{ggplot2} package \parencite{Wickham2016} to create visualizations.  
\end{examplebox}

\textit{Answer:}

\subsection{Which computational environment do you plan to use?}

\explanation{Please specify the operating system and its version which you intend to use. If the study is performed on multiple machines or servers, provide information for each one of them, if possible.}

\begin{examplebox}
We will run the simulation study on a Windows 11 machine. The complete output of \texttt{sessionInfo()} will be saved and reported in the supplementary materials.
\end{examplebox}

\textit{Answer:}

\subsection{Which other steps will you undertake to make simulation results reproducible? \textmd{(optional)}}

\explanation{This can include sharing the code and full or intermediate results of the simulation in an open online repository. Additionally, this may include supplemental materials or interactive data visualizations, such as a shiny application.}

\begin{examplebox}
We will upload the fully reproducible simulation script and a data set containing all relevant estimates, standard errors, and \textit{p}-values for each repetition of the simulation to OSF (\url{https://osf.io/dfgvu/}) and GitHub (\url{https://github.com/bsiepe/SimPsychReview}). 
\end{examplebox}

\textit{Answer:}

\subsection{Is there anything else you want to preregister? \textmd{(optional)}}

\explanation{For example, the answer could include the most likely obstacles in the simulation design, and the plans to overcome them, or measures that increase the trust in the preregistration date (e.g., setting the seed based on a future event), as explained in the introduction of this template.}

\begin{examplebox}
No.
\end{examplebox}

\textit{Answer:}

\newpage
\section*{References}

\nocite{Siepe2024} \printbibliography[heading=none]

\textbackslash end\{document\}

\printbibliography

\end{document}
